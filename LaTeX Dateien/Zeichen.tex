\section{Abk�rzungsverzeichnis und Zeichenerkl�rungen}

\begin{tabular}{l @{$\qquad\qquad$} l}

$\mathbb{N}$
& Nat�rliche Zahlen\\

$\mathbb{N}_0$
& Nat�rliche Zahlen \emph{inklusive} 0\\

$\mathbb R$
& Reelle Zahlen\\

$\mathbb{R}^+$
& die Menge reeller, nicht negativer Zahlen\\

$F_X$
& Verteilungsfunktion der indizierten Zufallsgr��e $X$\\

$\mathbb{E}(X)$
& Erwartungswert der Zufallsgr��e $X$\\

$Var(X)$
& Varianz der Zufallsgr��e $X$\\

$\left(\Omega,\mathfrak{A},\mathbb{P}\right)$
& Wahrscheinlichkeitsraum\\

$X\sim F$
& die Zufallsgr��e $X$ habe die Verteilung $F$\\

$p_k$
& Wahrscheinlichkeitsfunktion diskreter Zufallsgr��en, $\mathbb{P}\left(X=k\right)$\\

$\mathds{1}$
& Identische Abbildung\\

$\#$ Menge
& Anzahl der Elemente in der Menge

\end{tabular}

%$X\sim F$
%& die Zufallsgr��e $X$ habe die Verteilung $F$\\
%
%$X \stackrel{D}{=} Y$
%& die Zufallsgr��en $X$ und $Y$ haben dieselbe Verteilung\\
%
%
%
%$\overline{F_X}$
%& Schwanzfunktion der Zufallsgr��e $X$\\
%
%$F_X^{-1}$
%& verallgemeinerte inverse Verteilungsfunktion der Zufallsgr��e $X$\\
%
%$F\star G$
%& Faltung der Verteilungsfunktionen $F$ und $G$\\
%
%$F^{\star n}$
%& n-fache Faltung der Verteilungsfunktion $F$\\
%
%$\EW(X)$
%& Erwartungswert der Zufallsgr��e $X$\\
%
%$\EW(X\mid Y)$
%& bedingter Erwartungswert der Zufallsgr��e $X$ gegeben $Y$\\
%
%$p_k$
%& Wahrscheinlichkeitsfunktion diskreter Zufallsgr��en, $\Pro\left(X=k\right)$\\
%
%$supp(F)$
%& Tr�ger der Verteilung $F$\\
%
%$\lim_{x\downarrow t}f(x)$
%& rechtsseitiger Grenzwert der Funktion $f$ an der Stelle $t$\\
%
%$\lim_{x\uparrow t}f(x)$
%& linksseitiger Grenzwert der Funktion $f$ an der Stelle $t$\\
%
%$f(t+)$
%& rechtsseitiger Grenzwert der Funktion $f$ an der Stelle $t$\\
%
%$f(t-)$
%& linksseitiger Grenzwert der Funktion $f$ an der Stelle $t$\\
%
%$\binom{n}{k}$
%& Binomialkoeffizient\\
%
%$\exp\left\{t\right\}$
%& Exponentialfunktion an der Stelle $t$\\
%
%$\det A$
%& Determinante der Matrix A\\
%
%$D$
%& Derivationsoperator\\
%
%$\mathbb{I}_M$
%& Indikatorfunktion auf der Menge $M$\\
%
%\end{tabular}
%
%\newpage
%
%\free
%
%\begin{tabular}{l @{$\qquad\qquad$} l}
%
%$\pi_X$
%& Stop-Loss-Transformierte der Zufallsgr��e $X$\\
%
%$\mathfrak X$
%& Menge der beschr�nkten Zufallsvariablen \\
%& \quad mit Erwartungswert $\mu$ und Varianz $\sigma^2$\\
%
%$\mathfrak{F_X}$
%& Menge der Verteilungsfunktionen der Zufallsvariablen aus $\mathfrak{X}$\\
%
%$\psi(u)$
%& Ruinwahrscheinlichkeit des klassischen Risikoprozesses \\
%& \quad bei einer Anfangsrisikoreserve $u$\\
%
%$\hat{L}_f$
%& Laplace-Transformierte der Funktion $f$\\
%
%$\hat{l}_f$
%& Laplace-Stieltjes-Transformierte der Funktion $f$\\
%
%$X\leq_{st} Y$
%& $Y$ dominiert $X$ im Bezug auf die stochastische Dominanz\\
%
%$X\leq_{D} Y$
%& $Y$ dominiert $X$ im Bezug auf die Dangerousness-Ordnung\\
%
%$X\leq_{sl} Y$
%& $Y$ dominiert $X$ im Bezug auf die Stop-Loss-Ordnung\\
%
%$F_X\preceq_{st} F_Y$
%& falls gilt $X\sim F_X$ und $Y\sim F_Y$, so dominiert\\
%& $Y$ die Zufallsgr��e $X$ im Bezug auf die stochastische Dominanz\\
%
%$F_X\preceq_{D} F_Y$
%& falls gilt $X\sim F_X$ und $Y\sim F_Y$, so dominiert\\
%& $Y$ die Zufallsgr��e $X$ im Bezug auf die Dangerousness-Ordnung\\
%
%$F_X\preceq_{sl} F_Y$
%& falls gilt $X\sim F_X$ und $Y\sim F_Y$, so dominiert\\
%& $Y$ die Zufallsgr��e $X$ im Bezug auf die Stop-Loss-Ordnung\\