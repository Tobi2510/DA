\documentclass[a4paper,titlepage,oneside]{scrreprt}

% Paket zum �bersetzen
\usepackage{ngerman}

% Eingabe von Umlauten
\usepackage[latin1]{inputenc}

% Verwenden von T1 Fonts
\usepackage[T1]{fontenc}

\usepackage[pdftex]{graphicx}
\usepackage{array}
\usepackage[bottom]{footmisc} %Tabelle nicht unter Fu�noten
\usepackage{enumitem}
\usepackage{amsfonts, amsmath, amsthm, amssymb}
\usepackage{subfig} 
\usepackage{float}
\usepackage{listings}
\usepackage{tikz}
\usepackage{dsfont}

\usetikzlibrary{automata,arrows,positioning,calc}

\usepackage[linktocpage,bookmarks,bookmarksopen={false},bookmarksnumbered,pdfstartview={FitH},
	pdftitle={Diplomarbeit},
	pdfauthor={Tobias Riedel},
	pdfsubject={Modellierung eines Risiko�quivalents f�r isolierte Ereignisse und singul�re Ereignisketten in Kranken- und Lebensversicherungsbiographien},
	pdfcreator={TR},
	pdfproducer={TR},
	pdfkeywords={Diplomarbeit, Tobias Riedel, Risikotheorie}]{hyperref}

%Einr�cken Beginn Absatz
\setlength{\parindent}{0em} 

\newtheorem{defini}{Definition}[chapter]
\newenvironment{defi}
{\begin{defini} \ \newline}
{\end{defini}}

\newtheorem{lemmas}[defini]{Lemma}
\newenvironment{lemma}
{\begin{lemmas} \ \newline}
{\end{lemmas}}


\newtheorem{koro}[defini]{Korollar}
\newenvironment{korollar}
{\begin{koro} \ \newline}
{\end{koro}}

\newtheorem{theorem}[defini]{Theorem}
\newenvironment{THEOREM}
{\begin{theorem} \ \newline}
{\end{theorem}}

\newtheorem{satz}[defini]{Satz}
\newenvironment{Satz}
{\begin{satz} \ \newline}
{\end{satz}}

\newtheorem{folg}[defini]{Folgerung}
\newenvironment{folgerung}
{\begin{folg} \ \newline}
{\end{folg}}