\section{Verwendete Abfragen} \label{sec:scripts}
\subsection{SQL-Scripte}

\begin{itemize}
	\item Abfrage zur Eingrenzung der Unf�lle nach Aufnahme- und Entlassunggrund
\begin{lstlisting}[language=sql]
SELECT KHD.[PERSONEN_ID]
			,KHD.[FALL_ID]
			,KHD.[IK]
			,KHD.[DRG]
			,KHD.[LEISTUNG_VON]
			,KHD.[LEISTUNG_BIS]
			,KHD.[VERWEILDAUER]
			,KHD.[AUFNAHMEGRUND_301]
			,KHD.[ENTLASSUNGSGRUND_301] 
			,KHD.[FACHABTEILUNG_301]
			,KHD.[ZAHLBETRAG]
			,KH_ICD.[DIAGNOSEART]
			,KH_ICD.[ICD_PRIMAER]     
INTO [falldaten_unfaelle]
FROM
	(SELECT * FROM [Krankenhaus_Fall]
	WERE (AUFNAHMEGRUND_301 LIKE '__03%' OR 
		AUFNAHMEGRUND_301 LIKE '__02%' OR 
		AUFNAHMEGRUND_301 LIKE '__07%') 
	AND (ENTLASSUNGSGRUND_301 NOT LIKE '07%' AND 
		ENTLASSUNGSGRUND_301 NOT LIKE '11%'))
	AS KHD
INNER JOIN [Fall_Diagnose] AS KH_ICD
ON KHD.FALL_ID =KH_ICD.FALL_ID AND KHD.IK = KH_ICD.IK;
\end{lstlisting}
\end{itemize}

\pagebreak
\subsection{R-Scripte}
\begin{itemize}
	\item Abfrage zur Eingrenzung der Unf�lle nach Aufnahme- und Entlassunggrund
\begin{lstlisting}[basicstyle = \small, language=R]
## Test der Parameter 2010
data <- read.csv(file="UnfallData2010_final.csv", sep=';', dec=',')
t.test(data$ZAHLBETRAG, mu=922.851)

data <- read.csv(file="vers2010Detail.csv", sep=';', dec='.')
data <- data[data$Geschlecht == "W" & data$Alter>0 & data$Alter<26,]
wtd.t.test(data$AnzahlUnfaelle/data$Gewicht,0.002720257,data$Gewicht)

data <- read.csv(file="vers2010Detail.csv", sep=';', dec='.')
data <- data[data$Geschlecht == "W" & data$Alter>25 & data$Alter<66,]
wtd.t.test(data$AnzahlUnfaelle/data$Gewicht,0.000584902,data$Gewicht)

data <- read.csv(file="vers2010Detail.csv", sep=';', dec='.')
data <- data[data$Geschlecht == "W" & data$Alter>65,]
wtd.t.test(data$AnzahlUnfaelle/data$Gewicht,0.002820436,data$Gewicht)

data <- read.csv(file="vers2010Detail.csv", sep=';', dec='.')
data <- data[data$Geschlecht == "M" & data$Alter>0 & data$Alter<26,]
wtd.t.test(data$AnzahlUnfaelle/data$Gewicht,0.003829016,data$Gewicht)

data <- read.csv(file="vers2010Detail.csv", sep=';', dec='.')
data <- data[data$Geschlecht == "M" & data$Alter>25 & data$Alter<66,]
wtd.t.test(data$AnzahlUnfaelle/data$Gewicht,0.000972688,data$Gewicht)

data <- read.csv(file="vers2010Detail.csv", sep=';', dec='.')
data <- data[data$Geschlecht == "M" & data$Alter>65,]
wtd.t.test(data$AnzahlUnfaelle/data$Gewicht,0.001866967,data$Gewicht)

## Test der Parameter 2011
data <- read.csv(file="UnfallData2011_final.csv", sep=';', dec=',')
t.test(data$ZAHLBETRAG, mu=922.851)

data <- read.csv(file="vers2011Detail.csv", sep=';', dec='.')
data <- data[data$Geschlecht == "W" & data$Alter>0 & data$Alter<26,]
wtd.t.test(data$AnzahlUnfaelle/data$Gewicht,0.002720257,data$Gewicht)

data <- read.csv(file="vers2011Detail.csv", sep=';', dec='.')
data <- data[data$Geschlecht == "W" & data$Alter>25 & data$Alter<66,]
wtd.t.test(data$AnzahlUnfaelle/data$Gewicht,0.000584902,data$Gewicht)

data <- read.csv(file="vers2011Detail.csv", sep=';', dec='.')
data <- data[data$Geschlecht == "W" & data$Alter>65,]
wtd.t.test(data$AnzahlUnfaelle/data$Gewicht,0.002820436,data$Gewicht)

data <- read.csv(file="vers2011Detail.csv", sep=';', dec='.')
data <- data[data$Geschlecht == "M" & data$Alter>0 & data$Alter<26,]
wtd.t.test(data$AnzahlUnfaelle/data$Gewicht,0.003829016,data$Gewicht)

data <- read.csv(file="vers2011Detail.csv", sep=';', dec='.')
data <- data[data$Geschlecht == "M" & data$Alter>25 & data$Alter<66,]
wtd.t.test(data$AnzahlUnfaelle/data$Gewicht,0.000972688,data$Gewicht)

data <- read.csv(file="vers2011Detail.csv", sep=';', dec='.')
data <- data[data$Geschlecht == "M" & data$Alter>65,]
wtd.t.test(data$AnzahlUnfaelle/data$Gewicht,0.001866967,data$Gewicht)
\end{lstlisting}
\end{itemize}
