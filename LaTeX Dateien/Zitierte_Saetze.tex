\section{Zitierte Erkenntnisse und Nebenrechnungen} \label{sec:zitate}

\begin{satz}\label{sa:trafo} (Transformationssatz)\\
Seien U und V offene Teilmengen des $\mathbb{R}^n$ und sei weiterhin $T:U\rightarrow V$ ein Diffeomorphismus. Das ist die Funktion $f$ auf $V$ genau dann �ber V integrierbar, wenn die Funktion $(f\circ T)|det DT|$ �ber U integrierbar ist. Es gilt dann:
\begin{eqnarray*}
	\int_U f(T(x))|det(D*T(x))|dx = \int_V f(y) dy.
\end{eqnarray*}
Dabei ist $D$ die Jacobi-Matrix und $det DT(x)$ die Funktionaldeterminante von T. \\
Vergleiche \cite[Kapitel 9 - Abschnitt 1]{koenig04}.\\
\end{satz}

\begin{satz}\label{sa:monotoneKonv} (Satz von der monotonen Konvergenz)\\
Seien $(\Omega,\mathfrak{A},\mathbb{P})$ ein Wahrscheinlichkeitsraum und $\{X_n\}_{n\in \mathbb{N}}$ eine nicht negative, fast sicher monoton wachsende Folge von Zufallsvariablen, dann gilt f�r ihre Erwartungswerte
\begin{eqnarray*}
	\lim_{n\rightarrow \infty}\mathbb{E}(X_n) = \mathbb{E}(\lim_{n\rightarrow \infty} X_n).
\end{eqnarray*}
Das hei�t Integration und Grenzwertbildung k�nnen vertauscht werden.\\
Vergleiche TODO:.\\
\end{satz}

\begin{nr} \label{nr:integral1} Wir zeigen mit Hilfe von vollst�ndiger Induktion, dass f�r alle $y_1...y_n \in \mathbb{R}$ und $t_0,t_1 \in \mathbb{R}$ gilt
\begin{eqnarray*}
	\int_{t_0}^{t_1}\int_{y_1}^{t_1}...\int_{y_{n-1}}^{t_1} dy_n...dy_1 = \frac{(t_1-t_0)^n}{n!}.
\end{eqnarray*}
Induktionsanfang:
\begin{eqnarray*}	 
	\int_{y_{n-1}}^{t_1} dy_n = (t_1-y_{n-1})
\end{eqnarray*}
Induktionsschritt: Sei
\begin{eqnarray*}	
	\int_{y_1}^{t_1}...\int_{y_{n-1}}^{t_1} dy_n...dy_2 = \frac{(t_1-y_1)^{n-1}}{(n-1)!}
\end{eqnarray*}
Dann ist
\begin{eqnarray*}	
	\int_{y_1}^{t_1}...\int_{y_n}^{t_1} dy_{n+1}...dy_2 &=& \int_{y_1}^{t_1} \frac{(t_1-y_2)^{n-1}}{(n-1)!} dy_2\\
																												&=& \big[ -\frac{1}{n!}(t-y_2)^n\big]_{y_1}^t \\
																												&=& \frac{1}{n!}(t-y_1)^n
\end{eqnarray*}
\qed
\end{nr}
