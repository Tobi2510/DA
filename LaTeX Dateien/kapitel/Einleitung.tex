\chapter{Einleitung}

Bei den Gesundheitsforen Leipzig wurde ein Modell entwickelt, welches aus Basis der Morbitit�tsinformationen eines Versicherten die H�he seiner, bis zum Lebensende, zu erwartenden Leistungskosten f�r die Krankenversicherung bestimmt. Dieses Modell bildet den thematischen Rahmen dieser Diplomarbeit. Das Leben eines Krankenversicherten wird in diesem Ansatz in drei Phasen unterteilt, welche von den titelgebenden isolierten Ereignissen und singul�ren Ereigniskette �berlagert werden. Das Ziel dieser Arbeit ist es ein Risiko�quivalent f�r diese beiden Modellbestandteile zu bestimmen.\\

Im zweiten Kapitel wird die Motivation f�r die Erstellung des Kohorten-Modells erl�utert und es wird in diesem Zusammenhang einen kurz Einblick in das deutsche Krankenversicherungssystem gegeben.  Anschlie�end werden die einzelnen Modellbestandteile vorgestellt und gegeneinander Abgegrenzt. Diese Zusammenfassung basiert auf den bisherigen Ver�ffentlichungen zum Kohorten-Modell:" `Leben und Tod- Spezifische Implikationen eines vermeintlich l�ngeren Lebens f�r die Versicherungswirtschaft"' (\cite{artikelVW1} und \cite{artikelVW1}) und "`Demographischer Wandel in Deutschland- Legenden und Mythen"' (\cite{artikelGesWirt}).\\   

Das dritte Kapitel stellt die ben�tigten mathematischen Grundlage f�r die Modellierung vor. Zu Beginn werden die in dieser Arbeit verwendeten Standardverteilungen eingef�hrt und die jeweiligen Eigenschaften gezeigt. Anschlie�end werden die Stochastische Prozesse, die in den sp�teren Modellen verwendet werden betrachtet. Dies betrifft in erster Linie den Poisson-Prozess und die Markov-Kette. \\

Im vierten und f�nften Kapitel werden dann die eigentlichen Modelle behandelt. Die vorgestellten mathematischen Methoden werden verwendet, um unter Ber�cksichtigung der Rahmenbedingung durch das Kohorten-Modell und den Gegebenheiten im deutschen Gesundheitswesen, jeweils einen Modellansatz zu konstruieren, der eine Prognose der erwarteten Leistungen erm�glicht. Im Anschluss wird darauf eingegangen wie isolierte Ereignisse und singul�re Ereignisketten in den Abrechnungsdaten der Krankenkassen erkannt werden k�nnen und gegen�ber den restlichen Leistungen abgrenzt werden sollten. Der Modellansatz f�r die isolierten Ereignisse wird schlie�lich am Beispiel von Gehirnersch�tterungen umgesetzt und die prognostizierten Kosten analysiert. F�r die singul�ren Ereignisketten wird lediglich eine systematische Beschreibung des Vorgehens bei der Umsetzung angegeben, da aufgrund der Anforderungen an die Datengrundlage eine konkrete Umsetzung nicht m�glich war.\\

In beiden Kapitel, wird versucht durch Beispiele aus der Praxis die verschiedenen Eigenschafte von isolierten Ereignissen und singul�ren Ereignisketten zu veranschaulichen. Diese Beispiel basieren auf den Erfahren aus verschiedenen Projekten bei den Gesundheitsforen Leipzig und Gespr�chen mit den Mitarbeitern.\\

Zum Abschluss der Arbeit werden die gewonnen Erkenntnisse zusammengefasst und ein Zusammenf�hrung der Resultate mit den Kohorten-Modell vorgenommen.