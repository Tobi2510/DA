\begin{titlepage}

\vspace{6cm}

\Huge
\textbf{Kurzzusammenfassung}

\vspace{2cm}

\normalsize
In der Welt der Krankenversicherung gibt es immer wieder die Frage, wie sich die zuk�nftigen Leistungskosten eines Versicherten absch�tzen lassen. Bei den Gesundheitsforen Leipzig ist zu diesem Zweck ein mathematisches Modell entstanden, dass diese Fragestellung beantworten soll: Das Kohorten-Modell. Bei diesem wird das Leben eines Versicherten in drei Phasen unterteilt, welche zus�tzlich von zwei Ereignisarten �berlagert werden. Die verschieden Lebensphasen wurden unter anderen in den Artikeln "`Leben und Tod- Spezifische Implikationen eines vermeintlich l�ngeren Lebens f�r die Versicherungswirtschaft"' (\cite{artikelVW1} und \cite{artikelVW1}) und "`Demographischer Wandel in Deutschland- Legenden und Mythen"' (\cite{artikelGesWirt}) ausf�hrlich diskutiert. Diese Diplomarbeit widmet sich deshalb der Beschreibung und Modellierung der �berlagernden Ereignisarten: Den isolierten Ereignissen und den singul�ren Ereignisketten. Es wird vorgef�hrt, welche mathematischen Methoden verwendet werden k�nnen, um geeignete Modelle zu konstruieren. Au�erdem werden die Besonderheiten im deutschen Gesundheitswesen erl�utert, die dabei beachtet werden m�ssen. Die vorgestellten Modelle werden dann praktisch und theoretisch implementiert und bewertet. Als Resultat kann ein Risiko�quivalent f�r dieses Teile des Kohorten-Modells angegeben werden. \\

\end{titlepage}

\begin{titlepage}

\vspace{6cm}

\Huge
\textbf{Danksagung}

\vspace{2cm}

\normalsize
An erster Stelle m�chte ich meinem Betreuer

\vspace{0.4cm}

PLATZHALTER2
\end{titlepage}